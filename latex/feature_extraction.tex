\subsection{Feature Extraction}
\subsubsection{Circularity, Elongation, Orientation}
Circularity, elongation, and orientation were used upon the recommendation of a few papers having success with these features in classifying characters and digits.  Circularity is useful in our analysis because many symbols have a circular shape or at least a circular component to their shape.  Elongation and orientation are useful to determine how stretched an image appears and what direction the image is rotated (if at all).  This is helpful because most people's handwriting is not perfect and is likely elongated or tilted.
\subsubsection{Euler Number}
The Euler number corresponding to an image is defined as the number of regions minus the number of holes in the region.  This is a very helpful feature to use when classifying symbols (digits and characters especially) because many have holes as part of their shape.
\subsubsection{Solidity, Perimeter, Equivalent Diameter, Eccentricity}
Solidity, perimeter, equivalent diameter, and eccentricity (along with a few others in this list) were taken as features mostly because they were directly pulled from the matlab \texttt{regionprops} command.  All of these can be used as more identifying features about a given region that may help distinguish one group of images from another.
\subsubsection{Mean, Standard Deviation, Skewness, and Kurtosis}
These four features are standard statistical measures of data that provide information about the average of the data set, how far the data is spread from the average, how skewed the data is from a normal distribution, and how pointed or flat the data set is compared to a normal distribution.  These were specifically run on the row and column data of the image to determine properties of the shape and relative position of objects as well as compare these features to other images.  
\subsubsection{Extent (Entire image vs. segments)}
Extent is a measure of the percentage of an image that is filled with illuminated pixels.  We used two different forms of this.  Extent was first calculated on the entire image to determine what percentage was filled.  This gives a relative measure of size that is homogeneous across images with different width and height but of the same character.  Next, the image was divided into a three by three grid and the percentage of illuminated pixels relative to the total amount of illuminated pixels was calculated for each region.  This was helpful in determining what portions of the image contained data and gave an overall idea of shape for a character.
