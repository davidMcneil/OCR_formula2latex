\section{Introduction}
%\textit{Document clearly describes the problem, Explicitly answers 3 questions in detail: Why is the problem interesting? Why is it challenging? What is interesting about the proposed solution?
%}

The problem of interpreting handwritten characters is not a new one; people have tackled the problem from different perspectives for quite a while now.  Our goal for this project was to create a system that could interpret handwritten text in images of mathematical formulas and convert them into LaTeX markup code for printing and display.  It's much more convenient and efficient to write out math equations rather than typing them, and many people are more comfortable writing out equations on paper or a whiteboard than attempting to setup and use LaTeX.  Not much mainstream software attempts to tackle this problem, and we thought it would be an interesting problem whose solution would be very useful.

\addImage{images/sample.jpg}{How can we take this handwritten formula and convert it to a computer friendly format?}{sample}{.35}

There are a couple challenging parts to this problem.  First of all, the sheer number of mathematical symbols used makes this problem quite daunting.  Because of the amount of symbols in use, finding training data to use to classify different symbols is equally challenging. Handwriting varies greatly between different people, and furthermore, many people don't write the same problem or even the same symbol in the same way every time.  Glare from reflective surfaces turned out to be a major complication as well.

Our solution to the problem involves the use of feature vectors and multi-classification SVMs, specifically a 15-class SVM.  These were used to classify images; based on a confidence value generated by each classifier in the SVM, the image was then classified based on which classifier claimed to be the most similar to the image.  Our method of extracting digits from a handwritten text image was rather unique as well. Using a combination of several different image processing techniques we were able extract all of the characters in an image.