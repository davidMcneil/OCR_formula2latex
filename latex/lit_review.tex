\section{Research}
%\textit{Document summarizes appropriate literature and shows that ideas from it were used in the project (revised from initial L.R.), Includes Reference section. 
%}

Optical character recognition (OCR) has been the subject of much research through the past and provides a direct application of machine learning. We were able to harness the work done in the past in order to enhance the work done in our project. 

Eikvil's work \cite{one} not only provided numerous methods of feature extraction but also summarized the robustness and practicality of each algorithm. Using sensitivity to noise, distortions, style, translation, and rotation to quantify robustness and speed, complexity, and independence to measure practicality. The paper strongly recommended using zoning and moments as features. Both of which we used in our project.

Hansen's work \cite{two} specifically outlined implementing OCR in Matlab which provided a general model for our implementation.

''Historical Review of OCR Research and Development" \cite{three} looks at two approaches to OCR: template matching, and structural analysis. The authors make the realization that these two approaches are overlapping and seem to be converging to a similar point. They propose a hybrid classification system using a combination of the two. We employed a similar technique by using zoning methods as well as properties of the image such as circularity or extent.